% Created 2012-12-16 Sun 03:37
\documentclass[11pt]{article}
\usepackage[utf8]{inputenc}
\usepackage[T1]{fontenc}
\usepackage{fixltx2e}
\usepackage{graphicx}
\usepackage{longtable}
\usepackage{float}
\usepackage{wrapfig}
\usepackage{soul}
\usepackage{textcomp}
\usepackage{marvosym}
\usepackage{wasysym}
\usepackage{latexsym}
\usepackage{amssymb}
\usepackage{hyperref}
\tolerance=1000
\providecommand{\alert}[1]{\textbf{#1}}

\title{User Manual}
\author{Bergar Simonsen}
\date{\today}
\hypersetup{
  pdfkeywords={},
  pdfsubject={},
  pdfcreator={Emacs Org-mode version 7.8.11}}

\begin{document}

\maketitle

\setcounter{tocdepth}{3}
\tableofcontents
\vspace*{1cm}
\section{Slice of Pie user manual}
\label{sec-1}
\subsection{Starting the application}
\label{sec-1-1}
\subsubsection{Running the application from Visual Studio}
\label{sec-1-1-1}

    Before starting the application, you need to start Visual Studio with administration priviledges.
    The reason for this is that the application will need to create files and folders for the documents
    and in order to do so the program needs to be run with administrator priviledges which will give the 
    application write access.

    Since the web client runs in the browser, the browser needs to allow pop up windows for localhost.
    This can be done when the application is run for the first time.

    When starting the application, you need to set the WebClient project as startup project (if it's not
    already set), and then run the program (f5 for debug mode, ctrl + f5 for the release version).

    The web client will start up with your default internet browser. 
    When the main page has loaded, you are ready to use the application.
\subsubsection{Signing up}
\label{sec-1-1-2}

    As a first time user there won't be any user registered in the system, so the first time you need
    to do is to sign up to use the system.

    To sign up, click the ``Sign up'' button. This will open up a new window with the sign up form.

    Fill out the form and click the ``Sign up'' button. A message will appear to say if the signup
    was successfull or not.

    If the signup was successfull you are now registered in the system, and are ready to use it. 
    Close the sign up window and go back to the main window.
\subsubsection{Logging in}
\label{sec-1-1-3}

    On the main screen there are two text boxes at the top of the window named ``username'' and ``password''.
    Enter the newly created username and password into the boxes and click the ``Login'' button.
\subsection{Using the system}
\label{sec-1-2}

   After you have logged in, press the ``Get Files'' button on the left page of the window. 
   This will show you all the files that belong to the current user. Since you are a new user you don't 
   have any documents, so you should only see the root folder (the one with your username).
\subsubsection{Create a new document.}
\label{sec-1-2-1}

    To create a new document, click the ``New Document''. This will clear all text boxes, and you are ready
    to write a new document.

    Creating a new document doesn't save the document, so before you go too far you should save your document.
    Write a filename in the filename box, and click the ``Save Document'' button.
    If you wish to save the document in a sub folder, just write the: ``foldername/filename.html'' in the filebox.

    The system doesn't require that you save the document as a HTML file, but the system is built around it. Not 
    doing so won't make it able to add images to you document.
\subsubsection{Deleting a document}
\label{sec-1-2-2}

    Deleting a document is very simple.
    Select a document from the list on the left. Make sure that the filename of the file is entered into the 
    filename box (this can also be done manually). Delete the document by clicking the ``Delete Document'' button.
\subsubsection{Sharing a document}
\label{sec-1-2-3}

    Sharing a document is very simple as well.
    Open the document you wish to share. Enter the username of the user you wish to share the document with in
    the text box next to the ``Share Document'' button, and click the share document button.
\subsubsection{Showing a document}
\label{sec-1-2-4}

    Since the document is built around the HTML format, the text area can't show any images or text formatting.
    In order to see the document (with images, formatting etc) you need to open it in another page in the 
    browser.
    Select the document you wish to view and click the ``Show Document'' button. This will open a new window 
    with your document in parsed HTML.

\end{document}
