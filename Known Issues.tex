% Created 2012-12-17 Mon 05:01
\documentclass[11pt]{article}
\usepackage[utf8]{inputenc}
\usepackage[T1]{fontenc}
\usepackage{fixltx2e}
\usepackage{graphicx}
\usepackage{longtable}
\usepackage{float}
\usepackage{wrapfig}
\usepackage{soul}
\usepackage{textcomp}
\usepackage{marvosym}
\usepackage{wasysym}
\usepackage{latexsym}
\usepackage{amssymb}
\usepackage{hyperref}
\tolerance=1000
\providecommand{\alert}[1]{\textbf{#1}}

\title{Known Issues}
\author{Bergar Simonsen}
\date{\today}
\hypersetup{
  pdfkeywords={},
  pdfsubject={},
  pdfcreator={Emacs Org-mode version 7.8.11}}

\begin{document}

\maketitle

\setcounter{tocdepth}{3}
\tableofcontents
\vspace*{1cm}
\section{Known Issues}
\label{sec-1}
\subsection{Web Client}
\label{sec-1-1}

\begin{itemize}
\item The File Tree doesn't update after a change has been made. The user has to click the 
     Get Files button every time a document is created or deleted.
\item Since the File Tree doesn't update itself it won't remove it self either. Clicking
     the Get Files button won't remove the existing File Tree but makes a new one. 
     The old one can then be ignored.
\item Saving a shared document isn't 100\% optimal. 
     When saving a shared document, it creates a new document in the current users 
     name. To allow the ``other'' user of the document to get the changes made to
     the shared document, you need to share it again, with the changes.
\item Every time a document is saved, and extra line break is added to each line. Since
     the document is based on HTML this has no effect on the outcome of the final 
     document, but can be annoying to look at.
\item Getting files sometimes takes longer than we anticipated.
\begin{itemize}
\item The reason for this is because of the shared documents. We waited till the end 
       to implement the shared documents. Because of this we had to make some ``hacks''
       in order to be done by the deadline.
       If we had more time this would be implemented otherwise.
\end{itemize}
\end{itemize}
\subsection{Stand Alone Client}
\label{sec-1-2}

\begin{itemize}
\item Creating files in the ``root-root'' folder if no folder is selected, the 
     ``create document'' method creates a document in a folder that should not 
     be accessable at all
\item ``Update settings'' creates new directory, even if user does not exist.
     When calling the method, it takes the ``username'' and creates a new folder 
     in currently selected directory
\item No way to log in on the standalone client, if your credentials are in the 
     ``username'' and ``password'' textboxes, this is considdered logging in.
\item When syncing, the document will not be successfull synched the first time
     and the user will not be told when is is done, only the server can see if the
     documents have been successfully synched.
\item When Synchonizing files that does not yet exist on the server, the server needs time 
     to create the files, this blocks the client from recieving the files, this results in
     the client only getting existing documents on server back.
\item Syncronize() should return a bool, describing wehter the sync was successfull or not
     most times when not successful it is because og above problem
\item It is only possible to synchronize once pr. session due to a bug.
\end{itemize}

\end{document}
