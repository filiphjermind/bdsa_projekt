% Created 2012-12-16 Sun 16:42
\documentclass[11pt]{article}
\usepackage[utf8]{inputenc}
\usepackage[T1]{fontenc}
\usepackage{fixltx2e}
\usepackage{graphicx}
\usepackage{longtable}
\usepackage{float}
\usepackage{wrapfig}
\usepackage{soul}
\usepackage{textcomp}
\usepackage{marvosym}
\usepackage{wasysym}
\usepackage{latexsym}
\usepackage{amssymb}
\usepackage{hyperref}
\tolerance=1000
\providecommand{\alert}[1]{\textbf{#1}}

\title{FinalReport}
\author{Bergar Simonsen}
\date{\today}
\hypersetup{
  pdfkeywords={},
  pdfsubject={},
  pdfcreator={Emacs Org-mode version 7.8.11}}

\begin{document}

\maketitle

\setcounter{tocdepth}{3}
\tableofcontents
\vspace*{1cm}
\section{Business Modeling}
\label{sec-1}
\subsection{Vision}
\label{sec-1-1}
\subsubsection{Introduction}
\label{sec-1-1-1}

Our goal is to make an interactive document sharing system, Slice of Pie,  which allows multiple users to easily share and edit documents both online and 
\subsubsection{Problem statement}
\label{sec-1-1-2}

Sharing and editing documents can be cumbersome. 
Sending a document back and forth between multiple users can lead to a lot of errors. Users can overwrite what another user has done, and if they aren't all
using the same text editing system this can lead to formatting issues in the document.
\subsubsection{Summary of system features}
\label{sec-1-1-3}

\begin{itemize}
\item Multiple users must be able to share and edit documents online.
\item Synchronization for offline usage.
\item Merging of documents.
\item History. Which allows the user to see all recent changes made to the document.
\item Documents can be categorized into folders or projects in order to get a better overview when working on a larger project with multiple files.
\end{itemize}
\subsection{Glossary}
\label{sec-1-2}

\begin{itemize}
\item \textbf{Response Time:} The time it takes for the system to respond to a request from the user.
\item \textbf{Document:} A document refers to a complete document, not just a single file. A document contains a owner, id, content and a file.
\item \textbf{Client:} A client can mean two things. A web client, operating directly on the server, and a stand-alone client, that runs offline, 
     locally on the end users machine synchronizing with the server.
\item \textbf{System:} Refers to the core of the application. This includes document handler, user handler etc ..
\end{itemize}
\section{Requirements / Analysis}
\label{sec-2}
\subsection{Use cases}
\label{sec-2-1}

\begin{itemize}
\item UC1: Create new document
\item UC2: Edit document
\item UC3: Delete document
\item UC4: Merging documents (resolve conflict)
\item UC5: Offline sync
\item UC6: New folder
\item UC7: New project
\item UC8: Find old version of document
\item UC9: Share Document
\item UC10: Log in
\end{itemize}
\subsubsection{Use case Diagram}
\label{sec-2-1-1}
\subsection{Supplementary Specification (FURPS+)}
\label{sec-2-2}

\begin{itemize}
\item Functionality
\begin{itemize}
\item The system must be able to create/edit/delete users.
\item The system must be able to create/edit/delete/share documents.
\item The system must keep a log of all document actions.
\item All system usage requires user authentication.
\item The system must support multiple users.
\end{itemize}
\item Usability
\begin{itemize}
\item The system must be easy to use.
\begin{itemize}
\item Have a clean user interface.
\item 8 out of 10 users must be able to use the system without any training.
\end{itemize}
\item The system must be easily visible for people with ``not perfect'' vision. 
       E.g no graphics that blurs the view of the core system functionality.
\item The web client must be easy and quick to navigate. No function should 
       be more than 3 clicks (windows/sub-windows) away.
\begin{itemize}
\item Not counting navigating a users files.
\end{itemize}
\end{itemize}
\item Reliability
\begin{itemize}
\item It must be possible to use the system without any internet connection.
\begin{itemize}
\item With some limitations.
\end{itemize}
\end{itemize}
\item Performance
\begin{itemize}
\item The system must respond instantly
\begin{itemize}
\item A request must not take more than a few (2-3) seconds.
\item Not taking external factors (such as bad internet connection) into account.
\end{itemize}
\end{itemize}
\end{itemize}
\subsection{Domain Model}
\label{sec-2-3}

   \textbf{INSERT DIAGRAM HERE}
\subsection{Logical Architecture}
\label{sec-2-4}

   \textbf{INSERT DIAGRAM HERE}
\subsection{System Sequence Diagram}
\label{sec-2-5}

   \textbf{INSERT DIAGRAM HERE}
\subsection{Operation Contracts}
\label{sec-2-6}

   \textbf{INSERT DIAGRAM HERE}
\section{Design}
\label{sec-3}
\subsection{Class Diagram}
\label{sec-3-1}

   \textbf{INSERT DIAGRAM HERE} 
\subsection{Interaction Diagrams}
\label{sec-3-2}

   \textbf{INSERT DIAGRAM HERE}  
\subsection{Technical Memos}
\label{sec-3-3}
\subsubsection{File Format}
\label{sec-3-3-1}

    \textbf{Issue:} Files format - Which file format to use
    \textbf{Solution:} Summary: Use HTML for our file format.
    \textbf{Factors:}
\begin{itemize}
\item Must be able to contain both text and
\end{itemize}
    \textbf{Solution:}
    We chose to use HTML for our file format because it's simple to construct, and can contain text and images seamlessly. 
    \textbf{Motivation:}
    We needed a file format that can contain images and text as well as being easy to construct. in addition, HTML can easily be extended to other content. 
    Lastly, HTML can be opened with any browser, so the users isn't tied to SliceOfPie if he just want's to view the content of a file.
    \textbf{Alternatives considered:}
    We considered using a .txt file format, but .txt can only contain plain text.
    We also considered using our own file format (since the format itself isn't important to the application). But if we use our own format the user is stuck
    with using SliceOfPie, so he can't view the content of a file with any other application.
\subsubsection{Document Merge}
\label{sec-3-3-2}

    \textbf{Issue:} Merging two versions of the same document.
    \textbf{Solution Summary:} Git-hub inspired merge.
    \textbf{Factors:}
\begin{itemize}
\item Merging two versions of the same document without overwriting existing changes.
\end{itemize}
    \textbf{Solution:}
    Our merging algorithm reads the two documents and stores them, line by line in an array. 
    Then the algorithm compares each line in the two arrays, if the lines are the same, insert the line into a new array. If the two lines aren't identical, 
    insert the new line into the new array + insert the line from the old array in the next line. This line will be encapsulated with $<<<$ TEXT $>>>$ which 
    shows the user where there is a conflict which the user can solve later on. 
    If the new version of the document contains lines that aren't in the old array, they are simply added to the new array. 
    \textbf{Motivation:}
    There are other, more advanced, merging algorithms available. Because of time constraint we chose to use this one. It isn't the most advanced/complete algorithm 
    but it does the job quite well considered it's simplicity.
    \textbf{Unresolved issues:}
\begin{itemize}
\item Our algorithm doesn't 100\% solve the conflict. In the end the user must manually chose which
      version to keep, and which version to discard.
\item If two identical lines exists in both versions but the lines is at another line number in the old
      document, this might cause a conflict $<<<$ TEXT $>>$ that could be avoided.
\end{itemize}
    \textbf{Alternatives considered:}
    An algorithm that analyses every line in the file keeps the one that the user wants.
\subsection{ER-Diagram}
\label{sec-3-4}

   \textbf{INSERT DIAGRAM HERE}
\subsection{User manual}
\label{sec-3-5}
\subsubsection{Starting the application}
\label{sec-3-5-1}
\begin{itemize}

\item Running the application from Visual Studio\\
\label{sec-3-5-1-1}%
Before starting the application, you need to start Visual Studio with administration priviledges.
     The reason for this is that the application will need to create files and folders for the documents
     and in order to do so the program needs to be run with administrator priviledges which will give the 
     application write access.

     Since the web client runs in the browser, the browser needs to allow pop up windows for localhost.
     This can be done when the application is run for the first time.

     When starting the application, you need to set the WebClient project as startup project (if it's not
     already set), and then run the program (f5 for debug mode, ctrl + f5 for the release version).

     The web client will start up with your default internet browser. 
     When the main page has loaded, you are ready to use the application.

\item Signing up\\
\label{sec-3-5-1-2}%
As a first time user there won't be any user registered in the system, so the first time you need
     to do is to sign up to use the system.

     To sign up, click the ``Sign up'' button. This will open up a new window with the sign up form.

     Fill out the form and click the ``Sign up'' button. A message will appear to say if the signup
     was successfull or not.

     If the signup was successfull you are now registered in the system, and are ready to use it. 
     Close the sign up window and go back to the main window.

\item Logging in\\
\label{sec-3-5-1-3}%
On the main screen there are two text boxes at the top of the window named ``username'' and ``password''.
     Enter the newly created username and password into the boxes and click the ``Login'' button.
\end{itemize} % ends low level
\subsubsection{Using the Application}
\label{sec-3-5-2}

    After you have logged in, press the ``Get Files'' button on the left page of the window. 
    This will show you all the files that belong to the current user. Since you are a new user you don't 
    have any documents, so you should only see the root folder (the one with your username).
\begin{itemize}

\item Create a new document.\\
\label{sec-3-5-2-1}%
To create a new document, click the ``New Document''. This will clear all text boxes, and you are ready
     to write a new document.

     Creating a new document doesn't save the document, so before you go too far you should save your document.
     Write a filename in the filename box, and click the ``Save Document'' button.
     If you wish to save the document in a sub folder, just write the: ``foldername/filename.html'' in the filebox.

     The system doesn't require that you save the document as a HTML file, but the system is built around it. Not 
     doing so won't make it able to add images to you document.

\item Deleting a document\\
\label{sec-3-5-2-2}%
Deleting a document is very simple.
     Select a document from the list on the left. Make sure that the filename of the file is entered into the 
     filename box (this can also be done manually). Delete the document by clicking the ``Delete Document'' button.

\item Sharing a document\\
\label{sec-3-5-2-3}%
Sharing a document is very simple as well.
     Open the document you wish to share. Enter the username of the user you wish to share the document with in
     the text box next to the ``Share Document'' button, and click the share document button.

\item Showing a document\\
\label{sec-3-5-2-4}%
Since the document is built around the HTML format, the text area can't show any images or text formatting.
     In order to see the document (with images, formatting etc) you need to open it in another page in the 
     browser.
     Select the document you wish to view and click the ``Show Document'' button. This will open a new window 
     with your document in parsed HTML.

\end{itemize} % ends low level
\section{Implementation}
\label{sec-4}
\section{Project Management}
\label{sec-5}

\end{document}
